\documentclass{article}
\usepackage[utf8]{inputenc}
\usepackage{graphicx}
\usepackage{caption}
\usepackage{subcaption}
\usepackage{listings}
\usepackage{xcolor}
\usepackage{placeins}
\usepackage{fancyhdr}

\newcommand\enunciat[2][blue]{\textcolor{#1}{\emph{#2}}}
\newcommand\subject{Tipologia i cicle de vida de les dades}
\newcommand\activity{PRAC1}

\title{\large \subject \\ \activity}
\author{Josep Alòs Pascual, Daniel Galan Vilella}
\date{\today}
\makeatletter

\pagestyle{fancy}
\lhead{\subject}
\rhead{\activity}
\rfoot{Josep Alòs Pascual, Daniel Galan Vilella}

\begin{document}
\maketitle

\section{Descripció del projecte}
\subsection{Context}
\enunciat{Explicar en quin context s'ha recol·lectat la informació. Explicar per
què el lloc web triat proporciona aquesta informació.}

\subsection{Descripció del dataset}
\enunciat{Definir un títol pel dataset. Triar un títol que sigui descriptiu.}

\enunciat{Descripció del dataset. Desenvolupar una descripció breu del conjunt
de dades que s'ha extret (és necessari que aquesta descripció tingui sentit amb
el títol triat).}

\subsection{Representació gràfica}
\enunciat{Presentar una imatge o esquema que identifiqui el dataset visualment}

\subsection{Contingut}
\enunciat{Explicar els camps que inclou el dataset, el període de temps de les
dades i com s'ha recollit.}

\subsection{Agraïments}
\enunciat{Presentar el propietari del conjunt de dades. És necessari incloure
cites de recerca o anàlisis anteriors (si n'hi ha).}

\subsection{Inspiració}
\enunciat{Explicar per què és interessant aquest conjunt de dades i quines
preguntes es pretenen respondre.}

\subsection{Llicència}
\enunciat{Seleccionar una d'aquestes llicències pel dataset resultant i explicar
el motiu de la seva selecció:
\begin{itemize}
    \item Released Under CC0: Public Domain License
    \item Released Under CC BY-NC-SA 4.0 License
    \item Released Under CC BY-SA 4.0 License
    \item Database released under Open Database License, individual contents
    under Database Contents License
    \item Other (specified above)
    \item Unknown License
\end{itemize}}


\section{Taula de contribucions}
\begin{table}
    \begin{tabular}{|l|l|}
        Contribucions & Signa \\\hline\hline
        Recerca prèvia & DGV, JAP \\\hline
        Redacció de les respostes & DGV, JAP \\\hline
        Desenvolupament del codi & DGV, JAP \\\hline
    \end{tabular}
    \caption{Taula de contribucions}
    \label{table:contribucions}
\end{table}
\end{document}
