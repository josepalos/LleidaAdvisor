\documentclass{article}
\usepackage[utf8]{inputenc}
\usepackage{graphicx}
\usepackage{caption}
\usepackage{subcaption}
\usepackage{listings}
\usepackage{xcolor}
\usepackage{placeins}
\usepackage{fancyhdr}

\newcommand\enunciat[2][blue]{\textcolor{#1}{\emph{#2}}}
\newcommand\subject{Tipologia i cicle de vida de les dades}
\newcommand\activity{PRAC1}

\title{\large \subject \\ \activity}
\author{Josep Alòs Pascual, Daniel Galan Vilella}
\date{\today}
\makeatletter

\pagestyle{fancy}
\lhead{\subject}
\rhead{\activity}
\rfoot{Josep Alòs Pascual, Daniel Galan Vilella}

\begin{document}
\maketitle

\section{Descripció del projecte}
\subsection{Context}
\enunciat{Explicar en quin context s'ha recol·lectat la informació. Explicar per
què el lloc web triat proporciona aquesta informació.}

Ens trobem en els inicis de la nostra vida laboral, on el nostre poder adquisitiu és superior que en la nostra època d'estudiants de grau. Cansats de freqüentar els mateixos bars i restaurants autoanomenats 'low cost', ens hem començat a interessar en restaurants d'un altre tipus. 

Utilitzant la pàgina de TripAdvisor hem anat explorant nous restaurants. Tot i això, com a bons informàtics, no som amics de les tasques manuals repetititves. 
Aprofitant aquesta pràctica, hem decidit que podem solucionar aquest problema mitjançant les tècniques d'analisi de dades, que estem aprenent en aquest master, de forma que ens pugui recomanar restaurants nous sense haver de pensar nosaltres on anar.

TripAdvisor és l'opció ideal per obtenir dades sobre restaurants ja que té informació sobre (1) la ubicació, (2) el preu aproximat, (3) comentaris i puntuacions, (4) i detalls de la seva cuina.


\subsection{Descripció del dataset}
\enunciat{Definir un títol pel dataset. Triar un títol que sigui descriptiu.}

Restaurants Lleida

\enunciat{Descripció del dataset. Desenvolupar una descripció breu del conjunt
de dades que s'ha extret (és necessari que aquesta descripció tingui sentit amb
el títol triat).}

El dataset estarà separat en dos arxius. En el primer s'inclourà la informació
referent als restaurants, i el segon sobre els comentaris dels usuaris. S'ha
decidit separar els dos conjunts per tal de mantenir la consistència dins d'un
mateix arxiu i evitar tenir dades duplicades.

\paragraph{Dataset 1: restaurants}. Aquest dataset conté informació sobre els
restaurants. Les dades i el seu tipus es mostren en la
taula~\ref{table:dataset1_data}.

\begin{table} % TODO review
    \begin{tabular}{|l|l|l|}
        Nom       & Tipus  & Descripció \\\hline\hline
        Nom       & String & Nom del restaurant \\\hline
        Adreça    & String & Direcció del restaurant \\\hline
        Telèfon   & String & Telèfon del restaurant \\\hline
        Puntuació & float  & La puntuació que té el restaurant. Aquesta
                             puntuació és la mitjana de les puntuacions de
                             menjar, servei, preu, i atmosfera. El seu valor
                             va d'1 a 5, i pot ser un nombre decimal. \\\hline
        Puntuació menjar & float & \\\hline
        Puntuació servei & float & \\\hline
        Puntuació preu & float & \\\hline
        Puntuació atmòsfera & float & \\\hline
        Horaris setmanals & ?? & Horaris que obra el restaurant. \\\hline
        Detalls cuina & string & ?? \\\hline
        Certificat excelència & boolean & Si el restaurant té un certificat
                                          d'excelència. \\\hline
    \end{tabular}
    \caption{Dades de restaurants.}
    \label{table:dataset1_data}
\end{table}

\paragraph{Dataset 2: comentaris}. % TODO
taula~\ref{table:dataset1_data}.

\begin{table} % TODO review
    \begin{tabular}{|l|l|l|}
        Nom            & Tipus  & Descripció \\\hline\hline
        Restaurant     & string & \\\hline
        Usuari         & string & \\\hline
        Titol          & string & \\\hline
        Text           & string & \\\hline
        Data de visita & date   & La data de la visita (mes i any) \\\hline
        Puntuació      & int    & \\\hline
        Resposta       & string & Resposta del restaurant \\\hline
    \end{tabular}
    \caption{Dades dels comentaris.}
    \label{table:dataset2_data}
\end{table}

\subsection{Representació gràfica}
\enunciat{Presentar una imatge o esquema que identifiqui el dataset visualment}

\subsection{Contingut}
\enunciat{Explicar els camps que inclou el dataset, el període de temps de les
dades i com s'ha recollit.}

\subsection{Agraïments}
\enunciat{Presentar el propietari del conjunt de dades. És necessari incloure
cites de recerca o anàlisis anteriors (si n'hi ha).}

\subsection{Inspiració}
\enunciat{Explicar per què és interessant aquest conjunt de dades i quines
preguntes es pretenen respondre.}

\subsection{Llicència}
\enunciat{Seleccionar una d'aquestes llicències pel dataset resultant i explicar
el motiu de la seva selecció:
\begin{itemize}
    \item Released Under CC0: Public Domain License
    \item Released Under CC BY-NC-SA 4.0 License
    \item Released Under CC BY-SA 4.0 License
    \item Database released under Open Database License, individual contents
    under Database Contents License
    \item Other (specified above)
    \item Unknown License
\end{itemize}}


\section{Taula de contribucions}
\begin{table}
    \begin{tabular}{|l|l|}
        Contribucions & Signa \\\hline\hline
        Recerca prèvia & DGV, JAP \\\hline
        Redacció de les respostes & DGV, JAP \\\hline
        Desenvolupament del codi & DGV, JAP \\\hline
    \end{tabular}
    \caption{Taula de contribucions}
    \label{table:contribucions}
\end{table}
\end{document}
